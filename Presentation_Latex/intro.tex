\section{Introduction}
\begin{frame}{Why Lightweight Cryptography?}
\begin{itemize}
    \item Lightweight Cryptography become one of the very important field from last few years, provides security with very low computation power
    \pause
    \item Deals with RFID tags, wireless sensor network, small internet enabled application that are called \textbf{IoT} arises
    \pause
    \item Should have less amount of code and easy to implement on hardware.
\end{itemize}
\end{frame}

\begin{frame}{Lightweight Block Ciphers}
    \begin{itemize}
        \item There are two structure of block ciphers
        \begin{itemize}
            \item Substitution Permutation Network(SPN)
            \item Feistel Network
        \end{itemize}
        \pause
        \item Feistel Network have same encryption and decryption process but inefficient in terms of number of round, ex. DES
        \pause
        \item SPN are efficient in terms of number of round but hard to implement because have different encryption and decryption process. That's leads to
        \pause
        \begin{itemize}
            \item big circuits for hardware, needs more space and energy
            \pause
            \item Too much code for smart card, makes it infeasible
            \pause
            \item For Software, cipher and it's inverse needs different code
        \end{itemize}
    \end{itemize}
\end{frame}





\begin{frame}{Loong : A LightWeight Block Cipher}
    \begin{itemize}
        \item Loong is a LightWeight Block Cipher that uses SPN structure
        \pause
        \item Loong uses SPN construction that have same encryption and decryption process
    \end{itemize}
    \pause
     Lets see how ...
\end{frame}